\chapter{Introdução}\label{cap:introducao}
\section{Contexto}
Estudar e entender o funcionamento do cérebro é de grande interesse, devido à sua complexidade. Além das técnicas tradicionais de estudos em neurociência, simulações computacionais dos neurônios e suas conexões são de grande importância, daí o surgimento da neurociência computacional, voltado a auxiliar na compreensão dos mecanismos cerebrais utilizando o computador. Por isso, este trabalho propõe a criação de um roteiro para um curso que possa introduzir a área de neurociência computacional para alunos de graduação. Além da introdução teórica, a parte prática também é apresentada, utilizando uma linguagem de programação livre para a simulação de diversos comportamentos observados no cérebro.

Alguns trabalhos com introduções semelhantes são encontrados na literatura. Dayan e Abbott~(\citeyear{dayan_theoretical_2001}) apresentam uma extensa revisão teórica dos conteúdos-chave de neurociências. Ermentrout e Terman~(\citeyear{ermentrout_mathematical_2010}) apresentam uma base voltada para os fundamentos matemáticos. Miller~(\citeyear{miller_introductory_2018}) apresenta uma introdução à neurociência computacional, porém utilizando o Matlab como ferramenta de programação, que é um \textit{software} pago. Um diferencial deste trabalho é o uso de ferramentas livres, podendo ser executadas gratuitamente, inclusive online.

\section{Objetivos}
\subsection{Objetivo Geral}
Elaborar um roteiro para execução de um curso introdutório em neurociências computacionais usando linguagens de programação livres.

\subsection{Objetivos Específicos}
\begin{itemize}
\item Obter uma bibliografia robusta para servir de base na elaboração do roteiro;
\item Criar um conjunto de códigos contendo exemplos dos conceitos apresentados ao longo do roteiro;
\item Consolidar o material criado em uma estrutura de fácil uso por interessados no tema em questão.
\end{itemize}

\section{Metodologia}
Um extenso referencial teórico é utilizado para apresentação dos diversos tópicos do curso. Na parte prática, códigos em Python, que implementam as teorias apresentadas, são disponibilizados e apresentados, com ênfase na relação entre os trechos de código e as partes teóricas associadas. A execução dos códigos é feita utilizando uma ferramente \textit{online} e gratuita, possibilitando a execução do curso sem qualquer tipo de instalação associada.

\section{Estrutura do Trabalho}
O trabalho está estruturado da seguinte maneira: o Capítulo~\ref{cap:teoria} mostra os elementos da base teórica apresentada. Uma breve apresentação de definições sobre neurobiologia, equações diferenciais ordinárias, probabilidade, noções sobre algoritmos e linguagem de programação são mostradas. O Capítulo~\ref{cap:modelos} fala dos modelos de neurônios de disparo mais conhecidos. O Capítulo~\ref{cap:conexoes} fala sobre as conexões entre neurônios, incluindo aprendizado e plasticidade de longa duração. O Capítulo~\ref{cap:ia} mostra as redes neurais e neuromórficas. Finalmente, o Capítulo~\ref{cap:conclusoes} contém as conclusões do trabalho e seus possíveis desdobramentos.
