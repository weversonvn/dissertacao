\chapter{Introdução}\label{cap:introducao}
\section{Contexto}

\section{Justificativa}

\section{Objetivos}
\subsection{Objetivo Geral}
Elaborar um roteiro para execução de um curso introdutório em neurociências computacionais usando linguagens de programação livres

\subsection{Objetivos Específicos}
\begin{itemize}
\item Obter uma bibliografia robusta para servir de base na elaboração do roteiro
\item Criar um conjunto de códigos contendo exemplos dos conceitos apresentados ao longo do roteiro
\item Consolidar o material criado em uma estrutura de fácil uso por interessados no tema em questão
\end{itemize}

\section{Metodologia}

\section{Estrutura do Trabalho}
O trabalho está estruturado da seguinte maneira: o Capítulo~\ref{cap:teoria} mostra os elementos da base teórica apresentada. Uma breve apresentação de definições sobre neurobiologia, equações diferenciais ordinárias, probabilidade, noções sobre algoritmos e linguagem de programação são mostradas. O Capítulo~\ref{cap:modelos} fala dos modelos de neurônios mais conhecidos. O Capítulo~\ref{cap:conexoes} fala sobre as conexões entre neurônios, incluindo aprendizado e plasticidade de longa duração. O Capítulo~\ref{cap:ia} mostra as redes neurais e neuromórficas. O Capítulo~\ref{cap:eeg} apresenta uma breve análise de sinais de Eletroencefalograma. Finalmente, o Capítulo~\ref{cap:conclusoes} contém as conclusões acerca do trabalho e os desdobramentos possíveis para este.
