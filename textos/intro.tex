% ----------------------------------------------------------
% Introdução
% ----------------------------------------------------------
\chapter{Introdução}\label{cap:introducao} % note que a divisão é em capítulos, e dentro dele que é em seções

\section{Contexto} % com exceção desta, as próximas seções são uma estrutura recomendada para a introdução

% eu mantive essa figura para ilustrar como usar
% eu prefiro salvar minhas figuras em formato vetorial
% aqui eu usei pdf, não sei se é o melhor, mas tem também eps e svg
%\begin{figure}[!htb]
    % entre colchete é como aparece na lista de ilustrações
    % o título tem que aparecer em cima, e em baixo a fonte, que é obrigatória mesmo que seja do próprio autor
	%\caption[Título curto]{Título longo: \textnormal{descrição.}}
	%\label{fig:titulo_para_referencia}
	%\centering
	%\includegraphics[width=.8\textwidth]{figura.pdf} \\
	% eu referenciei minha figura assim, mas não é a regra
	% eu não lembro da norma dizer algo sobre esse formato
	%\begin{small}\textbf{Fonte: \cite{item_da_bibliografia}}\end{small}
%\end{figure}

% sintaxe tanto da nota de rodapé quanto para colocar url, que pode ser clicada no pdf
bla blah \footnote{link para o repositório com o código implementado: \href{https://weversonvn.github.io}{texto do url (é melhor colocar a própria url)}} balh.

% eu uso equação assim, que numera e da para referenciar no texto
% eu já vi texto que numera as equações de maneira corrida desde o começo do texto, aqui é por seção, não lembro da norma dizer algo sobre isso, mas de qualquer maneira aqui ta assim pq não sei como fazer do outro jeito :)
% é separado por ~ para não ter quebra de linha (vale também para fig, tab, etc)
bla blah na equação~\ref{eq:afim}.

\begin{equation}\label{eq:afim}
	\mathbf{y}=x+2\text{ texto sem formatação de equação}
\end{equation}

% caso alguém queira aqui ta como colocar um algoritmo
 Algoritmo~\ref{alg:execucao}.

\begin{algorithm}[H]
   \label{alg:execucao}
   \SetAlgoLined
   \Entrada{variáveis} 
   \Saida{retornos}
   \Inicio{
   \Para{classes}{
       \Para{amostras}{
           \Para{características}{
               treina\\
               testa\\
               calcula acurácia\\
           }
       }
       calcula acurácia\\
   }
   exibe os resultados graficamente
   }
   \caption{\textsc{Execução}}
\end{algorithm}

% para as tabelas eu criei arquivos .tex separados, porque algumas ficam muito grandes e atrapalham na leitura aqui
% deve-se usar includes também

% referências apareceção de maneira adequada no local de citação e no final
blah blah \cite{autor-titulo-ano} blah.

\section{Justificativa}

\section{Motivação}

\section{Objetivos}
\subsection{Objetivo Geral}
\begin{itemize}
\item ...
\end{itemize}

\subsection{Objetivos Específicos}
\begin{itemize}
\item ...
\item ...
\item ...
\end{itemize}

\section{Metodologia}

\section{Estrutura do Trabalho}
