\chapter{Base teórica}\label{cap:teoria}
\section{Introdução}\label{sec:teoria_intro}

\section{Neurobiologia básica}\label{sec:fisiologia}
\subsection{Propriedades elétricas dos neurônios}

\begin{description}
	\item[Potencial de membrana ($V_M$)] A diferença de potencial entre a parte interna e a externa da célula neuronal
	$$V_M=V_{dentro}-V_{fora}$$
	\item[Equação de Nernst ($E_A$)] O valor do potencial de membrana onde o fluxo de um íon particular está em equilíbrio (o fluxo de saída é igual ao fluxo de entrada). É chamado também de potencial de reversão
	$$
	E_A=\frac{k_BT}{z_Aq_e}ln\Big(\frac{A_{fora}}{A_{dentro}}\Big)
	$$
	sendo $A$ o íon, $z_A$ a carga do íon, $A_{fora}$ e $A_{dentro}$ a concentração desse íon fora e dentro da célula neuronal, respectivamente, $T$ a temperatura absoluta (em Kelvin), $k_B$ a a constante de \textit{Boltzmann} ($1,39*10^{-23}JK^{-1}$) e $q_e$ a carga elétrica fundamental ($1,6*10^{-19}C$)
	
	\begin{tabular}{ccccc}
		\hline
		Ion & Carga & Conc. Interna & Conc. Externa & Potencial de Nernst \\
		\hline
		Sódio & +1 & 15nM & 120nM & 61,6mV \\
		
		Potássio & + & 150nM & 6nM & -86,1mV \\
		
		Cálcio & +2 & 50nM & 2nM & 141,7mV \\
		\hline
	\end{tabular}
	
	% tabela de potenciais de equilíbrio
	\item[Potencial de repouso (de equilíbrio)] O valor do potencial de membrana onde o fluxo de corrente elétrica de todos os íons é equilibrado dentro e fora da célula neuronal (o potencial de membrana não se altera). O valor típico para um neurônio é próximo de $-70mV$
	\item[Canais iônicos] Canais proteicos na membrana da célula neuronal que permitem a movimentação de íons através deles. Podem ser de dois tipos: com ou sem portão. Canais sem portão estão sempre abertos, enquanto os com portão podem abrir ou fechar, dependendo do valor do potencial de membrana, e por isso são chamados de canais iônicos dependentes de tensão
	
	\begin{figure}[htb!]
		\centering
		\caption[Canais iônicos de potássio]{Canais iônicos de potássio. Em \textbf{a} os íons de potássio saem da célula, causando um excesso de cargas positivas fora e negativas dentro. Em \textbf{b} o fluxo para fora e dentro é igual, causando equilíbrio}
		\label{fig:canaisions}
		\includegraphics[width=0.7\linewidth]{figs/canais_ions}
		\\
		\cite{ermentrout_mathematical_2010}
	\end{figure}
	
	\item[Despolarização] Ocorre quando íons positivos (como $Na^+$) entram na célula neuronal, elevando o potencial de membrana para valores mais positivos, até próximo de 0
	\item[Hiperpolarização] Ocorre quando íons positivos (como $K^+$) saem da célula neuronal, ou negativos (como $Cl^-$) entram na célula neuronal, deixando o potencial de membrana cada vez mais negativo
	%%\item[Constante de tempo da membrana] 
	%\item[Modelos de único compartimento] 
\end{description}

\section{Equações diferenciais ordinárias}\label{sec:eqdif}
\begin{itemize}
	\item Equação diferencial ordinária: uma equação relacionando uma função desconhecida $y(t)$, algumas derivadas de $y(t)$ e a variável $t$, geralmente representando o tempo \cite{adkins_ordinary_2012}
	\item Ordem: a ordem da maior derivada que aparece na equação diferencial
	\item $t$: variável independente
	\item $y$: variável dependente (depende de $t$)
	\item A solução é uma família de equações, que depende da escolha de constantes
\end{itemize}

\begin{figure}[htb!]
	\centering
	\caption{Soluções $y(t) = t + 1 + ce^t$ da equação $y'=y-t$ para vários $c$}
	\label{fig:solucao}
	\includegraphics[width=0.7\linewidth]{figs/solucao}
\end{figure}


\subsection{Exemplos}
\subsubsection{Decaimento radioativo}
Segundo a lei do decaimento radioativo, a taxa na qual os átomos radioativos desintegram é proporcional ao número total de átomos radioativos presente. Sendo $N(t)$ o número de átomos radioativos no tempo $t$, então $N'(t)$ é a taxa de mudança. A lei do decaimento radioativo é a que segue:

$$N'(t) = -\lambda N(t)$$
onde $\lambda$ é a constante de decaimento.

\begin{figure}[htb!]
	\centering
	\caption{Decaimento radioativo ($\lambda = 0,5$)}
	\label{fig:decaimento}
	\includegraphics[width=0.7\linewidth]{figs/decaimento}
\end{figure}

\subsubsection{Equações de Lotka-volterra}
Também conhecidas como equações predador-presa, são um par de equações diferenciais de primeira ordem, frequentemente usadas para descrever a dinâmica de sistemas biológicos de interação entre duas espécies, uma como predadora e a outra como presa. As populações de cada uma das espécies são dadas pelo par de equações:

$$
x' = ax - bxy
$$$$
y' = dxy - cy
$$
onde:\\
$
x: \text{população da presa}\\
y: \text{população do predador}\\
x', y': \text{taxas de variação de cada população}\\
a, b, c, d: \text{parâmetros que descrevem a interação entre as espécies}
$

\begin{figure}[htb!]
	\centering
	\caption{Sistema de Lotka-Volterra ($a$ = 1,5; $b$ = 1; $c$ = 3; $d$ = 1)}
	\label{fig:lotka-volterra}
	\includegraphics[width=0.7\linewidth]{figs/lotka-volterra}
\end{figure}


\subsubsection{Trajetória pendular}
O pêndulo é um dispositivo que contém uma massa atrelada a um fio e que oscila em torno de um ponto fixo. A equação do movimento para o ângulo $\theta$ (o ângulo que o pêndulo faz com a vertical) é:

$$
\frac{d^2\theta}{dt^2} = -\frac{1}{Q}\frac{d\theta}{dt} + \sin{\theta} + d\cos{\Omega t}
$$
onde:\\
$
t: \text{tempo}\\
Q: \text{fator de qualidade}\\
d: \text{amplitude}\\
\Omega: \text{frequência}
$
\\\\
Como se trata de uma equação diferencial de segunda ordem, é necessária a redução para duas equações de primeira ordem. Fazendo a substituição de variáveis $\omega = \frac{d\theta}{dt}$ podemos reescrever da seguinte maneira:

$$
\frac{d\theta}{dt} = \omega
$$$$
\frac{d\omega}{dt} = -\frac{1}{Q}\omega + \sin{\theta} + d\cos{\Omega t}
$$

\begin{figure}[htb!]
	\centering
	\caption[Trajetória pendular]{Trajetória pendular ($Q$ = 2; $d$ = 1,5; $\Omega$ = 0,65)}
	\label{fig:pendulo}
	\includegraphics[width=0.7\linewidth]{figs/pendulo}
\end{figure}

\subsection{Método de Euler}
Equações diferenciais ordinárias podem ser resolvidas analiticamente (não abordado neste curso) ou numericamente. Dentre os vários métodos existentes para a solução numérica, a adotada aqui é o método de Euler. Considere a equação $\frac{dx}{dt}=f(x,t)$, com $f(x,t)$ uma função qualquer de $x$ em relação à $t$. Dado um valor inicial $x0$ (usualmente com $t=0$), é possível simular a equação usando pontos discretos com intervalos $\Delta t$ fixos. Cada valor $x_n$ é dado por $x_n=x(t_n=n\Delta t)$. A partir disso, é possível usar o método de Euler avançado para calcular um valor seguinte a partir do valor anterior, ou seja:
$$
x_{n+1}=x_n+f(x_n,t_n)\Delta t
$$

\begin{figure}[htb!]
	\centering
	\caption{Método de Euler}
	\label{fig:euler}
	\includegraphics[width=0.7\linewidth]{figs/euler}
\end{figure}

Outros métodos não abordados no curso incluem o método de Euler reverso e o Runge-Kutta de segunda e quarta ordens, que são mais precisos na solução.

\section{Probabilidade}\label{sec:probabilidade}
\begin{itemize}
	\item Experimento aleatório: pode fornecer resultados diferentes a cada vez que se repete da mesma maneira
	\item Espaço amostral (S): conjunto de resultados possíveis para um experimento aleatório (pode ser contínuo ou discreto)
	\item Espaço amostral discreto: conjunto finito ou infinito contável de resultados
	\item Espaço amostral contínuo: intervalo (finito ou infinito) de números reais
	\item Evento (E): subconjunto do espaço amostral
\end{itemize}


\subsection{Probabilidade}
\begin{itemize}
	\item Probabilidade: quantifica a chance de ocorrer o resultado de um experimento aleatório (“A chance de chover hoje é de 30\%")
	\item Axiomas:
	\begin{enumerate}
		\item $P(S)=1$
		\item $0\leq P(E)\leq 1$
		\item $E_1\cap E_2=\emptyset\to P(E_1)+P(E_2)$
	\end{enumerate}
	\item Probabilidade da união: $P(A\cup B)=P(A)+P(B)-P(A\cap B)$
	\item Probabilidade condicional: $P(B|A)=P(A\cap B)/P(A),\quad P(A)>0$
	\item Teorema de Bayes:  $P(A|B)=\frac{P(B|A)P(A)}{P(B)},\quad P(B)>0$\\ % ex. do teste de droga
	\begin{description}
		\item[Exemplo:] Pelo fato de um novo procedimento médico ter se mostrado efetivo na detecção prévia de uma doença, propôs-se um rastreamento médico da população. A probabilidade de o teste identificar corretamente alguém com a doença, dando positivo, é $0,99$, e a probabilidade de o teste identificar corretamente alguém sem a doença, dando negativo, é $0,95$. A incidência da doença na população em geral é $0,0001$. Você fez o teste e o resultado foi positivo. Qual é a probabilidade de você ter a doença?
		\item[Solução:] Seja $D$ o evento em que você tem a doença e seja $S$ o evento é que o teste é positivo. A probabilidade requerida pode ser denotada como $P(D|S)$. A probabilidade de o teste identificar corretamente alguém sem a doença, dando negativo, é $0,95$. Consequentemente a probabilidade de um teste positivo sem a doença é
		$$P(S|D') = 0,05$$
		Do Teorema de Bayes,
		\begin{align*}
			P(D|S)&=P(S|D)P(D)/[P(S|D)P(D)+P(S|D')P(D')]\\
			&=0,99*0,0001/[0,99*0,0001+0,05*(1-0,0001)]\\
			&=1/506=0,002
		\end{align*}
		\item[Interpretação Prática:] A probabilidade de você ter a doença da de um resultado positivo do teste é somente 0,002. Surpreendentemente, embora o teste seja efetivo, no sentido de que $P(S|D)$ é alto e $P(S|D')$ é baixo, por causa da incidência da doença na população em geral ser baixa, as chances são bem pequenas de você realmente ter a doença, mesmo se o teste for positivo
	\end{description}
\end{itemize}


\subsection{Variáveis aleatórias}
\begin{itemize}
	\item Variável aleatória ($X$): função que atribui um número real ($x$) a cada resultado no espaço amostral de um evento aleatório
	\item Discretas: número de pessoas adultas em um ambiente; numero de carros em uma rodovia
	\item Contínuas: corrente elétrica; temperatura; tempo
	\item Função densidade de probabilidade discretas:
	\begin{enumerate}
		\item $f(x_i) \geq 0$ (para todo $x$)
		\item $\sum_{i=1}^n f(x_i)=1$
		\item $f(x_i)=P(X=x_i)$
	\end{enumerate}
	\item Função de distribuição cumulativa discretas: $F(x)=P(X\leq x)=\sum_{x_i\leq x}f(x_i)$
	\begin{enumerate}
		\item $0\leq F(x)\leq 1$
		\item Se $x\leq y$, então $F(x)\leq F(y)$
	\end{enumerate}
	\item Função densidade de probabilidade contínuas:
	\begin{enumerate}
		\item $f(x) \geq 0$ (para todo $x$)
		\item $\int_{-\infty}^\infty f(x)\mathrm{d}x=1$
		\item $P(a\leq X\leq b)=\int_a^b f(x)\mathrm{d}x=$ área sob $f(x)$ de $a$ a $b$ para qualquer $a$ e $b$
	\end{enumerate}
	\item Função de distribuição cumulativa contínuas: $F(x)=P(X\leq x)=\int_{-\infty}^{x}f(u)\mathrm{d}u$
\end{itemize}
\subsubsection{Média e variância}
\begin{itemize}
	\item Média (valor esperado) de uma variável aleatória discreta: $\mu=E(X)=\sum_{x}xf(x)$
	\item Variância de uma variável aleatória discreta: $\sigma^2=V(X)=E(X-\mu)^2$ (desvio-padrão: $\sigma=\sqrt{\sigma^2}$)
	\item Média (valor esperado) de uma variável aleatória contínua: $\mu=E(X)=\int_{\infty}^{\infty}xf(x)\mathrm{d}x$
	\item Variância de uma variável aleatória contínua: $\sigma^2=V(X)=\int_{-\infty}^{\infty}x^2f(x)\mathrm{d}x-\mu^2$ (desvio-padrão: $\sigma=\sqrt{\sigma^2}$)
\end{itemize}

\subsubsection{Distribuição de Poisson}
$$
f(x)=\frac{e^{-\lambda T}(\lambda T)^x}{x!}, x=0,1,2,\dots
$$
\begin{itemize}
	\item $T$: intervalo do evento
	\item $\lambda$: número médio de eventos por intervalo ($0\leq\lambda$)
	\begin{description}
		\item[Exemplo:] Falhas ocorrem ao acaso ao longo do comprimento de um fio delgado de cobre. Suponha que o número de falhas siga a distribuição de Poisson, com uma média de 2,3 falhas por milímetro. Determine a probabilidade de existirem exatamente duas falhas em 1 milímetro de fio.
		\item[Solução:] Seja $X$ o número de falhas em 1 milímetro de fio. Então, $E(X)=2,3$ falhas e
		$$P(X=2) = \frac{e^{-2,3}(2,3)^2}{2!}=0,265$$
		Para determinar a probabilidade de 10 falhas em 5 milímetros de fio, consideramos $X$ o número de falhas em 5 milímetros de fio. Então, $X$ tem uma distribuição de Poisson com
		$$\lambda T=5\text{ mm X }2,3\text{ falhas/mm}=11,5\text{ falhas}$$
		Consequentemente,
		$$P(X=10)=e^{-11,5}\frac{(11,5)^{10}}{10!}=0,113$$
		\item[Interpretação Prática:] Dadas as suposições para um processo de Poisson e um valor para $\lambda$, as probabilidades podem ser calculadas para intervalos arbitrários de comprimento.
	\end{description}
\end{itemize}

\subsubsection{Distribuição normal (Gaussiana)}
$$
f(x)=\frac{1}{\sqrt{2\pi\sigma}}e^{\frac{-(x-\mu)^2}{2\sigma^2}}\qquad-\infty<x<\infty
$$

$$
E(X)=\mu\qquad V(X)=\sigma^2
$$

\begin{figure}[htb!]
	\centering
	\caption{Funções densidade de probabilidade normal para diferentes valores de $\mu$ e $\sigma^2$}
	\label{fig:normal}
	\includegraphics[width=0.7\linewidth]{figs/normal}
\end{figure}


\begin{itemize}
	\item Normal padrão: $\Phi(z)=P(Z\leq z)$, quando $\mu=0$ e $\sigma=1$
\end{itemize}


\section{Noções de algoritmos e programação}\label{sec:algoritmo}
\begin{itemize}
	\item \textbf{Algoritmo}: sequência de instruções para executar uma determinada tarefa. Ex.: algoritmo para lavar as mãos
	\begin{enumerate}
		\centering
		\item Início
		\item Abrir a torneira
		\item Molhar as mãos
		\item Ensaboar as mãos
		\item Molhar as mãos
		\item Secar as mãos
		\item Fim
	\end{enumerate}
	
	\item \textbf{Programa}: conjunto de instruções escritas em um arquivo com regras específicas
	\begin{verbatim}
		print("Olá, mundo!")
	\end{verbatim}
	\item \textbf{Linguagem de programação}: converte o programa escrito em ações no computador (ex.: \textit{Python}, \textit{C++}, \textit{Java})
\end{itemize}

\begin{figure}[htb!]
	\centering
	\caption{Do código para e/s}
	\label{fig:codigoio}
	\includegraphics[width=0.7\linewidth]{figs/codigo_io}
\end{figure}

\SetKwComment{Comment}{/* }{ */}
\begin{algorithm}
	\caption{Exemplo}\label{alg:ohm}
	\KwData{$n \geq 0$}
	\KwResult{$y = x^n$}
	$y \gets 1$\;
	$X \gets x$\;
	$N \gets n$\;
	\While{$N \neq 0$}{
		\eIf{$N$ é par}{
			$X \gets X \times X$\;
			$N \gets \frac{N}{2}$ \Comment*[r]{Este é um comentário}
		}{\If{$N$ é impar}{
				$y \gets y \times X$\;
				$N \gets N - 1$\;
			}
		}
	}
\end{algorithm}
