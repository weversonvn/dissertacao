\documentclass[
	% -- opções da classe memoir --
	12pt,				% tamanho da fonte
	openright,			% capítulos começam em pág ímpar (insere página vazia caso preciso)
	oneside,			% para impressão em frente e verso. Oposto a oneside
	a4paper,			% tamanho do papel.
	% -- opções da classe abntex2 --
	chapter=TITLE,		% títulos de capítulos convertidos em letras maiúsculas
	%section=TITLE,		% títulos de seções convertidos em letras maiúsculas
	%subsection=TITLE,	% títulos de subseções convertidos em letras maiúsculas
	%subsubsection=TITLE,% títulos de subsubseções convertidos em letras maiúsculas
	% -- opções do pacote babel --
	english,			% idioma adicional para hifenização
	french,				% idioma adicional para hifenização
	spanish,			% idioma adicional para hifenização
	brazil				% o último idioma é o principal do documento
	]{abntex2}

% ---
% Pacotes básicos 
% ---
\usepackage{lmodern}			% Usa a fonte Latin Modern
\usepackage{mathptmx}			% Usa a fonte Times New Roman
\usepackage[T1]{fontenc}		% Selecao de codigos de fonte.
\usepackage[utf8]{inputenc}		% Codificacao do documento (conversão automática dos acentos)
\usepackage{lastpage}			% Usado pela Ficha catalográfica
\usepackage{indentfirst}		% Indenta o primeiro parágrafo de cada seção.
\usepackage{color}				% Controle das cores
\usepackage{graphicx}			% Inclusão de gráficos
\usepackage{subcaption}				% Inclusão de gráficos lado a lado
\usepackage{microtype} 			% para melhorias de justificação
\usepackage{tabularx,ragged2e}	% Para inserir tabelas
\usepackage{multirow}			% Para mesclar células
\usepackage[dvipsnames,table,xcdraw]{xcolor}		% Permite adicionar cores nas linhas de tabelas
\usepackage{fancyvrb}			% Permite adicionar arquivos de texto
\usepackage[portuguese, ruled, linesnumbered]{algorithm2e} % Uso de algoritmos
\usepackage{amsfonts}			% Permite usar notação de conjuntos
\usepackage{amsmath}			% Permite citar equações
\usepackage{amsthm}				% Permite criar teoremas e experimentos
\usepackage{amssymb}            % Usei para o símbolo de transposto
\usepackage[font={bf, small}, labelsep=endash, labelfont=bf]{caption}	% Faz legenda de figuras ficarem em negrito
\usepackage[final]{pdfpages}
\usepackage{textcomp}           % para não dar erro no gensymb
\usepackage{gensymb}            % Para inserir o símbolo de grau em ângulos
\usepackage{enumitem}           % Para criar listas "numeradas" por letras (no abntex2 tem alineas no lugar)
\usepackage{bm,amsbsy}          % Para usar símbolos em negrito

\newcolumntype{L}{>{\RaggedRight\arraybackslash}X}
% ---

\usepackage[alf, abnt-emphasize=bf]{abntex2cite}	% Citações padrão ABNT
\usepackage{modelo-ufpa/ufpa}
\usepackage{minted}				% Para exibir código-fonte com realce de cores

% Muda o título de lista de ilustrações para lista de figuras
\addto\captionsbrazil{%
  \renewcommand{\listfigurename}%
    {Lista de Ilustrações}%
	\renewcommand{\listtablename}%
    {Lista de Tabelas}%
}

% Permite utilizar figuras sem precisar colocar o caminho absoluto
\graphicspath{{imagens/}}

% Define o ambiente de experimentos
\theoremstyle{definition}
\newtheorem{experimento}{Experimento}[section]
\newcommand{\experimentoautorefname}{Experimento}

%% Novo estilo
\makepagestyle{estilo_pretextual} %%% escolha um nome
  \makeevenhead{estilo_pretextual}{}{}{\ABNTEXfontereduzida \textbf \thepage}
  \makeoddhead{estilo_pretextual}{}{}{\ABNTEXfontereduzida \textbf \thepage}

%% Customiza comando \pretextual
\renewcommand{\pretextual}{
  \pagenumbering{roman} %%% ou \pagenumbering{Roman}
  \aliaspagestyle{chapter}{estilo_pretextual}% customizing chapter pagestyle
  \pagestyle{estilo_pretextual}
  \aliaspagestyle{cleared}{empty}
  \aliaspagestyle{part}{estilo_pretextual}
}

% ---
% Ajusta a marca \textual para que a numeração volte a ser arábica
% nos elementos textuais
\let\oldtextual\textual        % copia o comando \textual anterior para \oldtextual
\renewcommand{\textual}{%
  \oldtextual%
  \pagenumbering{arabic} % volta à numeração arábica
}
% ---


% ---
% Informações de dados para CAPA, FOLHA DE ROSTO e FICHA CATALOGRÁFICA
% ---
\universidade{Universidade Federal do Pará}
\instituto{Instituto de Tecnologia}
\faculdade{Programa de Pós-Graduação em Engenharia Elétrica}
\titulo{Nome do trabalho}
\autor{Weverson Vieira do Nascimento}
\local{Belém - PA}
\data{\the\year}
\tipotrabalho{Dissertação}
\grau{Mestre em Engenharia Elétrica } % manter o espaço no final
\preambulo{Dissertação apresentada como exigência parcial para obtenção do grau de \imprimirgrau pela \imprimiruniversidade. Área de concentração: Engenharias.}
\sobrenome{Nascimento}
\nome{Weverson Vieira do}
\palavraschave{
primeira.
segunda.
terceira.
quarta.
quinta.
}

% orientadores
\orientador{Prof. Dr. Fulano}
%\coorientador{Prof. Dr. Ciclano}
\datadadefesa{Data da Defesa: \underline{\qquad \qquad\qquad}}% janeiro de 2018}
\conceito{Conceito: \underline{\qquad\qquad\qquad}}
\primeiromembrodabanca{Prof.ª Dra. Ciclana}
\segundomembrodabanca{Prof. Dr. Fulano}
%\terceiromembrodabanca{Prof. Dr. Nome Sobrenome}
%\quartomembrodabanca{Prof. Dr. Outro Nome}
\faculdadecoordenador{Prof.ª Dra. Maria Emília de Lima Tostes}
% ---
% informações do PDF
\makeatletter
\hypersetup{
     	%pagebackref=true,
		pdftitle={\imprimirtitulo}, 
		pdfauthor={\imprimirautor},
    	pdfsubject={\imprimirpreambulo},
	    pdfcreator={LaTeX with abnTeX2},
		pdfkeywords={\imprimirpalavraschave}, 
		colorlinks=true,       		% false: boxed links; true: colored links
    	linkcolor=black,          	% color of internal links
    	citecolor=black,        		% color of links to bibliography
    	filecolor=magenta,      		% color of file links
		urlcolor=black,
		bookmarksdepth=4,
        breaklinks=true
}
\makeatother
% --- 

% --- 
% Espaçamentos entre linhas e parágrafos 
% --- 

% O tamanho do parágrafo é dado por:
\setlength{\parindent}{1.3cm}

% Controle do espaçamento entre um parágrafo e outro:
\setlength{\parskip}{0.2cm}  % tente também \onelineskip
\linespread{1.5} % espaçamento entre linhas

% ---
% compila o indice
% ---
\makeindex
% ---


% ----
% Início do documento
% ----
\begin{document}
\selectlanguage{brazil}
\frenchspacing 

% ----------------------------------------------------------
% ELEMENTOS PRÉ-TEXTUAIS
% ----------------------------------------------------------
% \pretextual

% ---
% Capa (obrigatório)
% ---
\imprimircapa
% ---

% ---
% Folha de rosto (obrigatório)
% ---
\imprimirfolhaderosto
% ---
% a ficha catalográfica oficial é a gerada pela biblioteca da instituição
% deve constar na versão final
% na ausência, usar a gerada pelo modelo
% \begin{fichacatalografica}
%     \includepdf{modelo-ufpa/ficha.pdf}
% \end{fichacatalografica}
\newpage
\begin{fichacatalografica}
	\imprimirfichacatalografica
\end{fichacatalografica}

% ---
% Inserir folha de aprovação (obrigatório)
% ---
% na versão final deve-se incluir a folha assinada
% \includepdf{folhadeaprovacao_final.pdf}
\begin{folhadeaprovacao}
	\ufpaPaginaDeAprovacao
    %\imprimirfolhadeaprovacao
\end{folhadeaprovacao}
% ---

% ---
% Dedicatória (opcional)
% ---
\begin{dedicatoria}
   \vspace*{\fill}
   \noindent
   \begin{flushright}
      \textit{escreva}\\
      \textit{aqui}\\
      \textit{a}\\
      \textit{dedicatória}
   \end{flushright}
\end{dedicatoria}
% ---

% ---
% Agradecimentos (opcional)
% ---
\begin{agradecimentos}
a escrever.
\end{agradecimentos}
% ---

% ---
% Epígrafe (opcional - NBR 10520)
% ---
\begin{epigrafe}
    \vspace*{\fill}
	\begin{flushright}
		\textit{``texto da epígrafe''\\
		autor}
	\end{flushright}
\end{epigrafe}
% ---

% ---
% RESUMOS
% ---
\setlength{\absparsep}{18pt}
\begin{resumo}
%O  resumo  deve  ressaltar  o  objetivo,  o  método,  os  resultados  e  as  conclusões  do  documento.  A  ordem  e  a  extensão destes itens dependem do tipo de resumo (informativo ou indicativo) e do tratamento que cada item recebe no documento original.
%O  resumo  deve  ser  composto  de  uma  sequência  de  frases  concisas,  afirmativas  e  não  de  enumeração  de  tópicos. Recomenda-se o uso de parágrafo único.
%A  primeira  frase  deve  ser  significativa,  explicando  o  tema  principal  do  documento.  A  seguir,  deve-se  indicar  a informação sobre a categoria do tratamento (memória, estudo de caso, análise da situação etc.).
texto do resumo

\textbf{Palavras-chave}: \imprimirpalavraschave
\end{resumo}

\begin{resumo}[Abstract]
 \begin{otherlanguage*}{english}
abstract

   \vspace{\onelineskip}
 
   \noindent 
   \textbf{Keywords}: escrever manualmente.
 \end{otherlanguage*}
\end{resumo}

% ---
% inserir lista de ilustrações (opcional)
% ---
\pdfbookmark[0]{\listfigurename}{lof}
\listoffigures*
\cleardoublepage
% ---

% ---
% inserir lista de quadros (não existe na norma, consta como ilustração)
% ---
%\pdfbookmark[0]{\listofquadrosname}{loq}
%\listofquadros*
%\cleardoublepage
% ---

% ---
% inserir lista de tabelas (opcional)
% ---
\pdfbookmark[0]{\listtablename}{lot}
\listoftables*
\cleardoublepage
% ---

% ---
% inserir lista de algoritmos (não consta na norma)
% ---
%\pdfbookmark[0]{\listalgorithmcfname}{loa}
%\imprimirlistadealgoritmos
%\cleardoublepage
% ---

% ---
% inserir lista de abreviaturas e siglas (opcional)
% ---
% listar em ordem alfabética
\begin{siglas}
\item[UFPA] Universidade Federal do Pará
\end{siglas}
% ---

% ---
% inserir lista de símbolos (opcional)
% ---
% inserir na ordem que aparecem no texto
\begin{simbolos}
\item[$\alpha$] Estrela de maior brilho de uma constelação % eu gosto de ver estrelas (literalmente)
\end{simbolos}
% ---

% ---
% inserir o sumario (obrigatório - NBR 6027)
% ---
\pdfbookmark[0]{\contentsname}{toc}
\tableofcontents*
\cleardoublepage
% ---

% ----------------------------------------------------------
% ELEMENTOS TEXTUAIS
% ----------------------------------------------------------
\textual

\chapter{Introdução}\label{cap:introducao}
\section{Contexto}
Estudar e entender o funcionamento do cérebro é de grande interesse, devido à sua complexidade. Além das técnicas tradicionais de estudos em neurociência, simulações computacionais dos neurônios e suas conexões são de grande importância, daí o surgimento da neurociência computacional, voltado a auxiliar na compreensão dos mecanismos cerebrais utilizando o computador. Por isso, este trabalho propõe a criação de um roteiro para um curso que possa introduzir a área de neurociência computacional para alunos de graduação. Além da introdução teórica, a parte prática também é apresentada, utilizando uma linguagem de programação livre para a simulação de diversos comportamentos observados no cérebro.

Alguns trabalhos com introduções semelhantes são encontrados na literatura. Dayan e Abbott~(\citeyear{dayan_theoretical_2001}) apresentam uma extensa revisão teórica dos conteúdos chave de neurociências. Ermentrout e Terman~(\cite{ermentrout_mathematical_2010}) apresentam uma base voltada para os fundamentos matemáticos. Miller~(\cite{miller_introductory_2018}) apresenta uma introdução à neurociência computacional, porém utilizando o Matlab como ferramenta de programação, que é um \textit{software} pago. Um diferencial deste trabalho é o uso de ferramentas livres, podendo ser executados gratuitamente, inclusive online.

\section{Objetivos}
\subsection{Objetivo Geral}
Elaborar um roteiro para execução de um curso introdutório em neurociências computacionais usando linguagens de programação livres

\subsection{Objetivos Específicos}
\begin{itemize}
\item Obter uma bibliografia robusta para servir de base na elaboração do roteiro
\item Criar um conjunto de códigos contendo exemplos dos conceitos apresentados ao longo do roteiro
\item Consolidar o material criado em uma estrutura de fácil uso por interessados no tema em questão
\end{itemize}

\section{Metodologia}
Um extenso referencial teórico é utilizado para apresentação dos diversos tópicos do curso. Na parte prática, códigos em Python, que implementam as teorias apresentadas, são disponibilizados e apresentados, com ênfase na relação entre os trechos de código e as partes teóricas associadas. A execução dos códigos é feita utilizando uma ferramente \textit{online} e gratuita, possibilitando a execução do curso sem qualquer tipo de instalação associada.

\section{Estrutura do Trabalho}
O trabalho está estruturado da seguinte maneira: o Capítulo~\ref{cap:teoria} mostra os elementos da base teórica apresentada. Uma breve apresentação de definições sobre neurobiologia, equações diferenciais ordinárias, probabilidade, noções sobre algoritmos e linguagem de programação são mostradas. O Capítulo~\ref{cap:modelos} fala dos modelos de neurônios de disparo mais conhecidos. O Capítulo~\ref{cap:conexoes} fala sobre as conexões entre neurônios, incluindo aprendizado e plasticidade de longa duração. O Capítulo~\ref{cap:ia} mostra as redes neurais e neuromórficas.
%O Capítulo~\ref{cap:eeg} apresenta uma breve análise de sinais de Eletroencefalograma.
Finalmente, o Capítulo~\ref{cap:conclusoes} contém as conclusões acerca do trabalho e os desdobramentos possíveis para este.

\chapter{Base teórica}\label{cap:teoria}
\section{Introdução}\label{sec:teoria_intro}

\section{Neurobiologia básica}\label{sec:fisiologia}
\subsection{Propriedades elétricas dos neurônios}

\begin{description}
	\item[Potencial de membrana ($V_M$)] A diferença de potencial entre a parte interna e a externa da célula neuronal
	$$V_M=V_{dentro}-V_{fora}$$
	\item[Equação de Nernst ($E_A$)] O valor do potencial de membrana onde o fluxo de um íon particular está em equilíbrio (o fluxo de saída é igual ao fluxo de entrada). É chamado também de potencial de reversão
	$$
	E_A=\frac{k_BT}{z_Aq_e}ln\Big(\frac{A_{fora}}{A_{dentro}}\Big)
	$$
	sendo $A$ o íon, $z_A$ a carga do íon, $A_{fora}$ e $A_{dentro}$ a concentração desse íon fora e dentro da célula neuronal, respectivamente, $T$ a temperatura absoluta (em Kelvin), $k_B$ a a constante de \textit{Boltzmann} ($1,39*10^{-23}JK^{-1}$) e $q_e$ a carga elétrica fundamental ($1,6*10^{-19}C$)
	
	\begin{tabular}{ccccc}
		\hline
		Ion & Carga & Conc. Interna & Conc. Externa & Potencial de Nernst \\
		\hline
		Sódio & +1 & 15nM & 120nM & 61,6mV \\
		
		Potássio & + & 150nM & 6nM & -86,1mV \\
		
		Cálcio & +2 & 50nM & 2nM & 141,7mV \\
		\hline
	\end{tabular}
	
	% tabela de potenciais de equilíbrio
	\item[Potencial de repouso (de equilíbrio)] O valor do potencial de membrana onde o fluxo de corrente elétrica de todos os íons é equilibrado dentro e fora da célula neuronal (o potencial de membrana não se altera). O valor típico para um neurônio é próximo de $-70mV$
	\item[Canais iônicos] Canais proteicos na membrana da célula neuronal que permitem a movimentação de íons através deles. Podem ser de dois tipos: com ou sem portão. Canais sem portão estão sempre abertos, enquanto os com portão podem abrir ou fechar, dependendo do valor do potencial de membrana, e por isso são chamados de canais iônicos dependentes de tensão
	
	\begin{figure}[htb!]
		\centering
		\caption[Canais iônicos de potássio]{Canais iônicos de potássio. Em \textbf{a} os íons de potássio saem da célula, causando um excesso de cargas positivas fora e negativas dentro. Em \textbf{b} o fluxo para fora e dentro é igual, causando equilíbrio}
		\label{fig:canaisions}
		\includegraphics[width=0.7\linewidth]{figs/canais_ions}
		\\
		\cite{ermentrout_mathematical_2010}
	\end{figure}
	
	\item[Despolarização] Ocorre quando íons positivos (como $Na^+$) entram na célula neuronal, elevando o potencial de membrana para valores mais positivos, até próximo de 0
	\item[Hiperpolarização] Ocorre quando íons positivos (como $K^+$) saem da célula neuronal, ou negativos (como $Cl^-$) entram na célula neuronal, deixando o potencial de membrana cada vez mais negativo
	%%\item[Constante de tempo da membrana] 
	%\item[Modelos de único compartimento] 
\end{description}

\section{Equações diferenciais ordinárias}\label{sec:eqdif}
\begin{itemize}
	\item Equação diferencial ordinária: uma equação relacionando uma função desconhecida $y(t)$, algumas derivadas de $y(t)$ e a variável $t$, geralmente representando o tempo \cite{adkins_ordinary_2012}
	\item Ordem: a ordem da maior derivada que aparece na equação diferencial
	\item $t$: variável independente
	\item $y$: variável dependente (depende de $t$)
	\item A solução é uma família de equações, que depende da escolha de constantes
\end{itemize}

\begin{figure}[htb!]
	\centering
	\caption{Soluções $y(t) = t + 1 + ce^t$ da equação $y'=y-t$ para vários $c$}
	\label{fig:solucao}
	\includegraphics[width=0.7\linewidth]{figs/solucao}
\end{figure}


\subsection{Exemplos}
\subsubsection{Decaimento radioativo}
Segundo a lei do decaimento radioativo, a taxa na qual os átomos radioativos desintegram é proporcional ao número total de átomos radioativos presente. Sendo $N(t)$ o número de átomos radioativos no tempo $t$, então $N'(t)$ é a taxa de mudança. A lei do decaimento radioativo é a que segue:

$$N'(t) = -\lambda N(t)$$
onde $\lambda$ é a constante de decaimento.

\begin{figure}[htb!]
	\centering
	\caption{Decaimento radioativo ($\lambda = 0,5$)}
	\label{fig:decaimento}
	\includegraphics[width=0.7\linewidth]{figs/decaimento}
\end{figure}

\subsubsection{Equações de Lotka-volterra}
Também conhecidas como equações predador-presa, são um par de equações diferenciais de primeira ordem, frequentemente usadas para descrever a dinâmica de sistemas biológicos de interação entre duas espécies, uma como predadora e a outra como presa. As populações de cada uma das espécies são dadas pelo par de equações:

$$
x' = ax - bxy
$$$$
y' = dxy - cy
$$
onde:\\
$
x: \text{população da presa}\\
y: \text{população do predador}\\
x', y': \text{taxas de variação de cada população}\\
a, b, c, d: \text{parâmetros que descrevem a interação entre as espécies}
$

\begin{figure}[htb!]
	\centering
	\caption{Sistema de Lotka-Volterra ($a$ = 1,5; $b$ = 1; $c$ = 3; $d$ = 1)}
	\label{fig:lotka-volterra}
	\includegraphics[width=0.7\linewidth]{figs/lotka-volterra}
\end{figure}


\subsubsection{Trajetória pendular}
O pêndulo é um dispositivo que contém uma massa atrelada a um fio e que oscila em torno de um ponto fixo. A equação do movimento para o ângulo $\theta$ (o ângulo que o pêndulo faz com a vertical) é:

$$
\frac{d^2\theta}{dt^2} = -\frac{1}{Q}\frac{d\theta}{dt} + \sin{\theta} + d\cos{\Omega t}
$$
onde:\\
$
t: \text{tempo}\\
Q: \text{fator de qualidade}\\
d: \text{amplitude}\\
\Omega: \text{frequência}
$
\\\\
Como se trata de uma equação diferencial de segunda ordem, é necessária a redução para duas equações de primeira ordem. Fazendo a substituição de variáveis $\omega = \frac{d\theta}{dt}$ podemos reescrever da seguinte maneira:

$$
\frac{d\theta}{dt} = \omega
$$$$
\frac{d\omega}{dt} = -\frac{1}{Q}\omega + \sin{\theta} + d\cos{\Omega t}
$$

\begin{figure}[htb!]
	\centering
	\caption[Trajetória pendular]{Trajetória pendular ($Q$ = 2; $d$ = 1,5; $\Omega$ = 0,65)}
	\label{fig:pendulo}
	\includegraphics[width=0.7\linewidth]{figs/pendulo}
\end{figure}

\subsection{Método de Euler}
Equações diferenciais ordinárias podem ser resolvidas analiticamente (não abordado neste curso) ou numericamente. Dentre os vários métodos existentes para a solução numérica, a adotada aqui é o método de Euler. Considere a equação $\frac{dx}{dt}=f(x,t)$, com $f(x,t)$ uma função qualquer de $x$ em relação à $t$. Dado um valor inicial $x0$ (usualmente com $t=0$), é possível simular a equação usando pontos discretos com intervalos $\Delta t$ fixos. Cada valor $x_n$ é dado por $x_n=x(t_n=n\Delta t)$. A partir disso, é possível usar o método de Euler avançado para calcular um valor seguinte a partir do valor anterior, ou seja:
$$
x_{n+1}=x_n+f(x_n,t_n)\Delta t
$$

\begin{figure}[htb!]
	\centering
	\caption{Método de Euler}
	\label{fig:euler}
	\includegraphics[width=0.7\linewidth]{figs/euler}
\end{figure}

Outros métodos não abordados no curso incluem o método de Euler reverso e o Runge-Kutta de segunda e quarta ordens, que são mais precisos na solução.

\section{Probabilidade}\label{sec:probabilidade}
\begin{itemize}
	\item Experimento aleatório: pode fornecer resultados diferentes a cada vez que se repete da mesma maneira
	\item Espaço amostral (S): conjunto de resultados possíveis para um experimento aleatório (pode ser contínuo ou discreto)
	\item Espaço amostral discreto: conjunto finito ou infinito contável de resultados
	\item Espaço amostral contínuo: intervalo (finito ou infinito) de números reais
	\item Evento (E): subconjunto do espaço amostral
\end{itemize}


\subsection{Probabilidade}
\begin{itemize}
	\item Probabilidade: quantifica a chance de ocorrer o resultado de um experimento aleatório (“A chance de chover hoje é de 30\%")
	\item Axiomas:
	\begin{enumerate}
		\item $P(S)=1$
		\item $0\leq P(E)\leq 1$
		\item $E_1\cap E_2=\emptyset\to P(E_1)+P(E_2)$
	\end{enumerate}
	\item Probabilidade da união: $P(A\cup B)=P(A)+P(B)-P(A\cap B)$
	\item Probabilidade condicional: $P(B|A)=P(A\cap B)/P(A),\quad P(A)>0$
	\item Teorema de Bayes:  $P(A|B)=\frac{P(B|A)P(A)}{P(B)},\quad P(B)>0$\\ % ex. do teste de droga
	\begin{description}
		\item[Exemplo:] Pelo fato de um novo procedimento médico ter se mostrado efetivo na detecção prévia de uma doença, propôs-se um rastreamento médico da população. A probabilidade de o teste identificar corretamente alguém com a doença, dando positivo, é $0,99$, e a probabilidade de o teste identificar corretamente alguém sem a doença, dando negativo, é $0,95$. A incidência da doença na população em geral é $0,0001$. Você fez o teste e o resultado foi positivo. Qual é a probabilidade de você ter a doença?
		\item[Solução:] Seja $D$ o evento em que você tem a doença e seja $S$ o evento é que o teste é positivo. A probabilidade requerida pode ser denotada como $P(D|S)$. A probabilidade de o teste identificar corretamente alguém sem a doença, dando negativo, é $0,95$. Consequentemente a probabilidade de um teste positivo sem a doença é
		$$P(S|D') = 0,05$$
		Do Teorema de Bayes,
		\begin{align*}
			P(D|S)&=P(S|D)P(D)/[P(S|D)P(D)+P(S|D')P(D')]\\
			&=0,99*0,0001/[0,99*0,0001+0,05*(1-0,0001)]\\
			&=1/506=0,002
		\end{align*}
		\item[Interpretação Prática:] A probabilidade de você ter a doença da de um resultado positivo do teste é somente 0,002. Surpreendentemente, embora o teste seja efetivo, no sentido de que $P(S|D)$ é alto e $P(S|D')$ é baixo, por causa da incidência da doença na população em geral ser baixa, as chances são bem pequenas de você realmente ter a doença, mesmo se o teste for positivo
	\end{description}
\end{itemize}


\subsection{Variáveis aleatórias}
\begin{itemize}
	\item Variável aleatória ($X$): função que atribui um número real ($x$) a cada resultado no espaço amostral de um evento aleatório
	\item Discretas: número de pessoas adultas em um ambiente; numero de carros em uma rodovia
	\item Contínuas: corrente elétrica; temperatura; tempo
	\item Função densidade de probabilidade discretas:
	\begin{enumerate}
		\item $f(x_i) \geq 0$ (para todo $x$)
		\item $\sum_{i=1}^n f(x_i)=1$
		\item $f(x_i)=P(X=x_i)$
	\end{enumerate}
	\item Função de distribuição cumulativa discretas: $F(x)=P(X\leq x)=\sum_{x_i\leq x}f(x_i)$
	\begin{enumerate}
		\item $0\leq F(x)\leq 1$
		\item Se $x\leq y$, então $F(x)\leq F(y)$
	\end{enumerate}
	\item Função densidade de probabilidade contínuas:
	\begin{enumerate}
		\item $f(x) \geq 0$ (para todo $x$)
		\item $\int_{-\infty}^\infty f(x)\mathrm{d}x=1$
		\item $P(a\leq X\leq b)=\int_a^b f(x)\mathrm{d}x=$ área sob $f(x)$ de $a$ a $b$ para qualquer $a$ e $b$
	\end{enumerate}
	\item Função de distribuição cumulativa contínuas: $F(x)=P(X\leq x)=\int_{-\infty}^{x}f(u)\mathrm{d}u$
\end{itemize}
\subsubsection{Média e variância}
\begin{itemize}
	\item Média (valor esperado) de uma variável aleatória discreta: $\mu=E(X)=\sum_{x}xf(x)$
	\item Variância de uma variável aleatória discreta: $\sigma^2=V(X)=E(X-\mu)^2$ (desvio-padrão: $\sigma=\sqrt{\sigma^2}$)
	\item Média (valor esperado) de uma variável aleatória contínua: $\mu=E(X)=\int_{\infty}^{\infty}xf(x)\mathrm{d}x$
	\item Variância de uma variável aleatória contínua: $\sigma^2=V(X)=\int_{-\infty}^{\infty}x^2f(x)\mathrm{d}x-\mu^2$ (desvio-padrão: $\sigma=\sqrt{\sigma^2}$)
\end{itemize}

\subsubsection{Distribuição de Poisson}
$$
f(x)=\frac{e^{-\lambda T}(\lambda T)^x}{x!}, x=0,1,2,\dots
$$
\begin{itemize}
	\item $T$: intervalo do evento
	\item $\lambda$: número médio de eventos por intervalo ($0\leq\lambda$)
	\begin{description}
		\item[Exemplo:] Falhas ocorrem ao acaso ao longo do comprimento de um fio delgado de cobre. Suponha que o número de falhas siga a distribuição de Poisson, com uma média de 2,3 falhas por milímetro. Determine a probabilidade de existirem exatamente duas falhas em 1 milímetro de fio.
		\item[Solução:] Seja $X$ o número de falhas em 1 milímetro de fio. Então, $E(X)=2,3$ falhas e
		$$P(X=2) = \frac{e^{-2,3}(2,3)^2}{2!}=0,265$$
		Para determinar a probabilidade de 10 falhas em 5 milímetros de fio, consideramos $X$ o número de falhas em 5 milímetros de fio. Então, $X$ tem uma distribuição de Poisson com
		$$\lambda T=5\text{ mm X }2,3\text{ falhas/mm}=11,5\text{ falhas}$$
		Consequentemente,
		$$P(X=10)=e^{-11,5}\frac{(11,5)^{10}}{10!}=0,113$$
		\item[Interpretação Prática:] Dadas as suposições para um processo de Poisson e um valor para $\lambda$, as probabilidades podem ser calculadas para intervalos arbitrários de comprimento.
	\end{description}
\end{itemize}

\subsubsection{Distribuição normal (Gaussiana)}
$$
f(x)=\frac{1}{\sqrt{2\pi\sigma}}e^{\frac{-(x-\mu)^2}{2\sigma^2}}\qquad-\infty<x<\infty
$$

$$
E(X)=\mu\qquad V(X)=\sigma^2
$$

\begin{figure}[htb!]
	\centering
	\caption{Funções densidade de probabilidade normal para diferentes valores de $\mu$ e $\sigma^2$}
	\label{fig:normal}
	\includegraphics[width=0.7\linewidth]{figs/normal}
\end{figure}


\begin{itemize}
	\item Normal padrão: $\Phi(z)=P(Z\leq z)$, quando $\mu=0$ e $\sigma=1$
\end{itemize}


\section{Noções de algoritmos e programação}\label{sec:algoritmo}
\begin{itemize}
	\item \textbf{Algoritmo}: sequência de instruções para executar uma determinada tarefa. Ex.: algoritmo para lavar as mãos
	\begin{enumerate}
		\centering
		\item Início
		\item Abrir a torneira
		\item Molhar as mãos
		\item Ensaboar as mãos
		\item Molhar as mãos
		\item Secar as mãos
		\item Fim
	\end{enumerate}
	
	\item \textbf{Programa}: conjunto de instruções escritas em um arquivo com regras específicas
	\begin{verbatim}
		print("Olá, mundo!")
	\end{verbatim}
	\item \textbf{Linguagem de programação}: converte o programa escrito em ações no computador (ex.: \textit{Python}, \textit{C++}, \textit{Java})
\end{itemize}

\begin{figure}[htb!]
	\centering
	\caption{Do código para e/s}
	\label{fig:codigoio}
	\includegraphics[width=0.7\linewidth]{figs/codigo_io}
\end{figure}

\SetKwComment{Comment}{/* }{ */}
\begin{algorithm}
	\caption{Exemplo}\label{alg:ohm}
	\KwData{$n \geq 0$}
	\KwResult{$y = x^n$}
	$y \gets 1$\;
	$X \gets x$\;
	$N \gets n$\;
	\While{$N \neq 0$}{
		\eIf{$N$ é par}{
			$X \gets X \times X$\;
			$N \gets \frac{N}{2}$ \Comment*[r]{Este é um comentário}
		}{\If{$N$ é impar}{
				$y \gets y \times X$\;
				$N \gets N - 1$\;
			}
		}
	}
\end{algorithm}

\chapter{Modelos de disparo de neurônio}\label{cap:modelos}
\section{Introdução}\label{sec:modelos_intro}

\section{Modelo integra-e-dispara com vazamento}\label{sec:modelolif}
% fig. circuito equivalente membrana

\begin{equation}
c_m\frac{\mathrm{d}V_m}{\mathrm{d}t}=G_{Na}(E_{Na}-V_m)+G_{Ca}(E_{Ca}-Vm)+G_K(E_K-V_m)+G_L(E_L-V_m)
\end{equation}

\subsection{Extenções do modelo LIF}

\subsubsection{Período refratário}

\subsubsection{Adaptação da taxa de disparo}

\subsection{Modelo LIF exponencial}

\subsection{Modelo LIF exponencial adaptativo}

\section{Modelo de Izhikevich}\label{sec:izhikevich}

\section{Modelo de Hodgkin-Huxley}\label{sec:modelohh}

\chapter{Conexões entre neurônios}\label{cap:conexoes}
\section{Introdução}\label{sec:conexoes_intro}

\section{Sinapses}\label{sec:sinapses}

\section{Sinapses dinâmicas}\label{sec:dinamicas}

\section{Biestabilidade e \textit{feedback} recorrente}\label{sec:biestabilidade}

\section{Geradores de padrão central}\label{sec:geradores}

\section{Circuitos tomadores de decisão}\label{sec:decisao}

\chapter{Aprendizado e plasticidade de longa duração}\label{cap:aprendizado}
\section{Introdução}\label{sec:aprendizado_intro}

\chapter{Inteligência artificial}\label{cap:ia} % ou aprendizado de máquina?
\section{Introdução}\label{sec:ia_intro}
A inteligência artificial é um amplo ramo que trata de sistemas capazes de aprender e interpretar informações, e usar esse aprendizado para objetivos específicos~\cite{haenlein_brief_2019}. Um dos campos da inteligência artificial é o aprendizado de máquina, onde a unidade de computação aprende a performar uma tarefa a partir de um conjuntos de exemplos de treinamento~\cite{louridas_machine_2016}, funcionando como um tipo de modelo computacional do cérebro. Esses modelos são construídos a partir da interconexão entre unidades de processamento, por vezes chamados de neurônios artificiais, e por isso são conhecidas como \textbf{redes neurais artificiais}.

O tipo mais comum de aprendizado de máquina é o chamado \textbf{aprendizado supervisionado}, onde o modelo é apresentado a um conjunto de dados de exemplo que são rotulados, ou seja, cada entrada tem a sua saída definida. Durante o aprendizado, o modelo fornece uma saída e, a partir da saída verdadeira, ele se ajusta internamente para se aproximar mais do real~\cite{lecun_deep_2015}. Quando não há uma saída real rotulada disponível, pode-se utilizar o \textbf{aprendizado não-supervisionado}. Nesse caso, o modelo tenta aprender alguns padrões nos dados, como grupos de exemplos similares. O aprendizado não-supervisionado tem a capacidade de identificar novos tipos de informações, já que não estão restritos aos rótulos~\cite{asnicar_machine_2023}.

Os algoritmos de aprendizado de máquina também podem ser divididos quanto à atividade a ser realizada. Atividades de \textbf{classificação} identificam a classe da informação (por exemplo, se um animal é um cão ou gato) a partir de dados rotulados, o \textbf{agrupamento} (\textit{clustering}, em inglês) determina classes agrupando informações similares em grupos (\textit{clusters}) rotulados (como o agrupamento de filmes em gêneros), a \textbf{regressão} é usada para prever algum valor ou quantidade (como o preço de ações ao longo do tempo), e a \textbf{redução de dimensionalidade} representa dados de várias dimensões em outras menores (como as várias informações de um paciente sendo reduzidas às mais importantes para o diagnóstico de uma doença).

\section{Redes neurais}\label{sec:redesneurais}
O primeiro modelo de neurônio artificial foi o de McCulloch-Pitts, baseado em lógica binária e que possuía duas fase, a primeira onde o neurônio soma as contribuições de todos os neurônios anteriores, e a segunda onde o neurônio dispara apenas se um limiar for ultrapassado e não houver disparo de neurônio inibitório~\cite{mcculloch_logical_1943}. Apesar de revolucionário para a época, um dos principais problemas do modelo de McCulloch-Pitts era a ausência do ajuste de pesos, ou seja, as conexões entre os neurônios sempre tinham a mesma força. Muitas pesquisas se desenvolveram na área para suprir essa necessidade, até que Frank Rosenblatt, inspirado pela teoria de Hebb vista anteriormente, criou o \textit{Perceptron}~\cite{rosenblatt_perceptron_1958},desenvolvido para ser um neurônio mais generalizado, sendo a base do aprendizado de máquina moderno. Matematicamente, o \textit{Perceptron} pode ser modelado pela seguinte equação:
\begin{equation}\label{eq:perceptron}
	y=\sigma\Big(\sum_{k=1}^Nw_kx_k+b\Big)
\end{equation}
onde $x_k$ é a resposta da sinapse $k$, $w_k$ é o peso da sinapse, $b$ é o \textit{bias} (viés, em português), um peso de referência que representa um conhecimento a priori, %TODO: elaborar
$\sigma$ é uma função de ativação não-linear, e $y$ é a saída do neurônio, como representado na Figura~\ref{fig:perceptron}. Resolvido o problema da atualização dos pesos, o \textit{Perceptron} sozinho é incapaz de realizar tarefas de separação não-linear~\cite{minsky_perceptrons_2017}, levando à criação, muitos anos depois, do \textit{Perceptron} de multi-camadas (MLP, \textit{Multilayer perceptron}, em inglês), que supre as limitações do \textit{Perceptron} simples ao agrupar camadas de neurônios, como na Figura~\ref{fig:mlp}. As camadas presentes entre a entrada e a saída da rede são chamadas de camadas ocultas (ou intermediárias).
\begin{figure}
	\centering
	\caption{Arquiteturas do Perceptron}
	\label{fig:perceptrons}
	\begin{subfigure}[b]{0.3\textwidth}
		\includegraphics[width=\textwidth]{figs/perceptron}
		\caption{Perceptron}
		\label{fig:perceptron}
	\end{subfigure}
	\qquad\qquad
	\begin{subfigure}[b]{0.3\textwidth}
		\includegraphics[width=\textwidth]{figs/mlp}
		\caption{Perceptron de multi-camadas}
		\label{fig:mlp}
	\end{subfigure}
	%TODO: trocar figura
\end{figure}

Os vários neurônios presentes nessa arquitetura frequentemente têm seus pesos atualizados através do algoritmo de gradiente descendente, que busca minimizar o erro entre a saída obtida pela rede e a real (aprendizado supervisionado), e essa atualização é feita através da metodologia da retro-propagação (\textit{backpropagation}, em inglês), ou seja, as atualizações das camadas finais da rede são propagadas em direção às iniciais~\cite{werbos_beyond_1974}. Quando há uma grande quantidade de camadas na rede ela é chamada de rede neural profunda (\textit{deep neural network}, em inglês), permitindo um mapeamento de dados mais complexos, o que não seria muito eficiente em redes não profundas. Uma lista não exaustiva de arquiteturas, profundas ou não, de redes neurais artificiais (\textit{artificial neural networks} (ANN), em inglês) é apresentada abaixo:
\begin{alineas}
	\item \textbf{\textit{Convolutional Neural Networks} (CNN, Redes neurais convolucionais)}: possui camadas que implementam a operação matemática de convolução, aplicando um filtro no sinal, e camadas de sub-amostragem (\textit{downsampling}), chamadas de \textit{pooling}. Essas redes lidam bem com sinais em duas dimensões, e por isso são frequentemente aplicadas em tarefas de visão computacional (relacionadas a imagens e vídeos)~\cite{lecun_gradient-based_1998};
	\item \textbf{\textit{Recurrent Neural Networks} (RNN, Redes neurais recorrentes)}: muito usadas em tarefas de entradas sequenciais, como processamento de fala e linguagem, possem a característica de processar os dados elemento a elemento, considerando também informações dos elementos passados, armazenados em pesos de conexões recorrentes, que funcionam como uma memória dinâmica para a rede~\cite{elman_finding_1990};
	\item \textbf{\textit{Hopfield Networks} (Redes Hopfield)}: proposta por J. J. Hopfield, são redes que são treinadas para aprender padrões (memórias) dos dados de maneira associativa, baseado no princípio de Hebb de que neurônios que disparam juntos se conectam;
	%TODO: citação hopfield
	\item \textbf{\textit{Autoencoder} (auto codificador)}: redes que aprendem a representar os dados reduzindo sua dimensionalidade, diminuindo as camadas ocultas, e, posteriormente, recriando os dados originais~\cite{hinton_reducing_2006};
	\item \textbf{\textit{Long Short-Term Memory} (LSTM, memória longa de curto prazo)}: uma variação da RNN composta de 4~(quatro) blocos de memória, sendo um principal e os outros três, chamados de portões de entrada, esquecimento e saída, responsáveis por alterar o estado da célula como um todo~\cite{hochreiter_long_1997}.	Devida à sua capacidade de reter informações de longo tempo em séries temporais, são muito usadas em sistemas de reconhecimento de voz;
\end{alineas}
Além das arquiteturas citadas acima, existem outras que são baseadas no princípio de funcionamento do neurônio, envolvendo, inclusive, alterações físicas dos equipamentos onde são implementadas. Essas redes são chamadas de \textbf{neuromórficas}, detalhadas na seção seguinte.

\section{Redes neuromórficas}\label{sec:redesneuromorficas}
A maioria das redes neurais convencionais, derivadas da MLP, são implementadas com base na arquitetura de computadores de Von Neumann, onde as unidades de memória e processamento ficam separadas~\cite{von_neumann_first_1993}. Algumas características dessa arquitetura incluem a codificação da informação para sinais binários (0 e 1) e os programas escritos em instruções explícitas (algoritmos), que são computadas pelo processador e os dados são armazenados na memória, trafegando por um barramento que conecta os dois elementos.

As arquiteturas neuromórficas surgem como uma opção onde estrutura e funcionamento são mais semelhantes ao do cérebro, em especial os neurônios e sinapses~\cite{ivanov_neuromorphic_2022}.As redes neurais desenvolvidas com base nos princípios dessa arquitetura são chamadas de \textbf{redes neurais de disparo}~(\textit{spiking neural networks}, SNN, em inglês), cuja arquitetura é mostrada na Figura~\ref{fig:snn}. Diferentemente da arquitetura de Von Neumann, a neuromórfica distribui o processamento e memória em conjunto nos elementos de neurônio e sinapses, respectivamente, evitando o tempo da transmissão de informação através do barramento entre memória e processador~\cite{indiveri_memory_2015}. Outra diferença da arquitetura neuromórfica é que a informação é codificada, tanto espacial quanto temporalmente, em sequências de potenciais de ação~\cite{frenkel_bottom-up_2023}. Além disso, os programas de computadores neuromórficos são definidos pela estrutura da rede neural e seus parâmetros~\cite{schuman_opportunities_2022}.
\begin{figure}[tb]
	\centering
	\caption[Arquitetura de redes neurais de disparo (SNN)]{Arquitetura de redes neurais de disparo (SNN)}
	\label{fig:snn}
	\includegraphics[width=0.7\linewidth]{figs/snn}
\end{figure}


Alguns projetos de dispositivos de \textit{hardware} foram desenvolvidos com base na arquitetura neuromórfica. O projeto \textbf{\textit{SpiNNaker}} propôs um computador paralelo massivo de milhões de núcleos, desenvolvido para a modelagem de redes neurais de disparo de grande escala~\cite{furber_spinnaker_2014}. O \textbf{\textit{BrainScaleS}} é um robusto sistema integrado de \textit{hardware} neuromórfico capaz de ser usado para modelar a complexidade neural, bem como topologias de rede realistas, permitindo a execução direta de modelos que anteriormente só eram possíveis de serem simulados numericamente~\cite{schemmel_wafer-scale_2010}. O projeto responsável pela sua criação, chamado de FACETS, participou do desenvolvimento do modelo AELIF, visto anteriormente, e esse modelo foi implementado fisicamente como um \textit{chip} presente no \textit{BrainScaleS}~\cite{aamir_lif_2017}. O projeto \textbf{\textit{TrueNorth}} criou um chip com mais de 5~(cinco) bilhões de transistores e 4096~(quatro mil e noventa e seis) núcleos sinápticos, interconectados por uma rede de chips internos que integra 1~(um) milhão de neurônios de disparo programáveis e 256~(duzentas e cinquenta e seis) milhões de sinapses configuráveis~\cite{merolla_million_2014}. O \textbf{\textit{Loihi}} é um chip de $60\ mm^2$ fabricado com processadores da Intel de $14-nm$ que integra uma variedade de recursos para o campo de redes neurais de disparo, como conectividade, compartimentos dendríticos, atrasos sinápticos e, mais importante, regras de aprendizado sinápticos programáveis~\cite{davies_loihi_2018}.

Diversos outros chips neuromórficos existem ou estão sendo pesquisados, com diferentes tecnologias de desenvolvimento e modelagem aplicadas~\cite{mehonic_brain-inspired_2022}, 
alguns deles utilizando uma tecnologia chamada de \textbf{memristor}, uma espécie de transistor capaz de alterar a sua condutividade dependendo da corrente ou tensão aplicada aos seus terminais, sendo úteis para a implementação eficiente de computação em memória para as redes neurais de disparo~\cite{mehonic_memristorsmemory_2020}. Vale destacar, também, que muitos desses dispositivos podem ser usados com redes neurais convencionais, como RNNs e CNNs, fazendo com que os estudos desse tipo de arquitetura estejam em alta.

As redes neurais de disparo usam, principalmente, neurônios do tipo integra-e-dispara para trocar informações via disparos de potencial de ação, e as unidades de computação são conectadas entre si e interagem através das sinapses~\cite{roy_towards_2019}. É possível converter as redes neurais convencionais em redes de disparo fazendo alguns ajustes de pesos e normalização, a fim de tirar proveito das vantagens proporcionados pelo uso do \textit{hardware} neuromórfico~\cite{diehl_conversion_2016}. Para fazer a conversão entre as redes neurais artificiais e as redes neurais de disparo, os dados de entrada precisam ser codificados em disparos de potencial de ação, e essa codificação pode ser baseada na taxa de disparo, na ordem de disparos de uma população, ou temporal.

Na codificação de taxa, a média temporal (quantidade de disparos por intervalo de tempo) ou a média de disparos de diferentes \textit{trials} é usada, tendo um correlato com o que ocorre em populações de neurônios do córtex auditivo primário~\cite{decharms_primary_1996}. Na codificação de ordem de disparos de uma população vários neurônios diferentes disparam um potencial de ação, porém o importante não é o instante exato em que eles disparam, e sim a ordem deles, ou seja, qual neurônio disparou primeiro, qual foi o segundo, e assim por diante~\cite{mallot_coding_2013}. A codificação temporal considera o exato momento em que ocorre o disparo do potencial de ação de cada neurônio, um comportamento neuronal de relevância~\cite{bohte_evidence_2004}. Subclassificações dessa codificação podem considerar apenas o instante do primeiro disparo ou a latência (a diferença de tempo entre os disparos). Os critérios para seleção do método de codificação variam por diferentes aspectos, como a minimização da perda de informação após a decodificação ou o aumento da acurácia de previsão/classificação.

Do ponto de vista do aprendizado das redes neurais de disparo, o não-supervisionado frequentemente faz uso da STDP como parte do seu mecanismo, com os pesos de potencialização e depressão de longa duração sendo atualizados durante o aprendizado~\cite{kheradpisheh_stdp-based_2018}. No caso do aprendizado supervisionado, existe uma variação do método \textit{backpropagation}, relacionando as ativações das unidades de redes neurais com as taxas de disparo~\cite{diehl_fast-classifying_2015}. No entanto, há discussões acerca do treinamento direto com \textit{backpropagation} na simulação do cérebro, principalmente na conversão de ANNs para SNNs. Algumas alternativas são encontradas na literatura, como o \textit{SpikeProp}, semelhante ao \textit{backpropagation}~\cite{bohte_error-backpropagation_2002}, o \textit{ReSuMe}, que pode ser usado em tarefas de tomada de decisão~\cite{ponulak_supervised_2010}, e o \textit{SPAN}, que transforma as sequências de potenciais de ação na fase de treinamento em sinais analógicos, para que operações matemáticas comuns possam ser aplicadas~\cite{mohemmed_span_2012}.

% pensar onde colocar os simuladores
% \section{Simuladores neuronais}\label{sec:simuladores}

%\chapter{Análise de sinais de eletroencefalograma}\label{cap:eeg}

% ----------------------------------------------------------
% Considerações Finais
% ----------------------------------------------------------
\chapter{Considerações finais}\label{cap:conclusoes}
\section{Conclusões}

\section{Trabalhos futuros}

% ---

% ----------------------------------------------------------
% ELEMENTOS PÓS-TEXTUAIS
% ----------------------------------------------------------
\postextual
% ----------------------------------------------------------

% ----------------------------------------------------------
% Referências bibliográficas (obrigatório - NBR 6023)
% ----------------------------------------------------------
% Os elementos essenciais são: autor(es), título, edição, local, editora e data de publicação
%  Quando se tratar de obras consultadas online,  também  são  essenciais  as  informações  sobre  o  endereço  eletrônico, apresentado  entre  os  sinais  <  >,  precedido  da  expressão  Disponível  em:  e  a  data  de  acesso  ao  documento,  precedida  da expressão Acesso em:, opcionalmente acrescida dos dados referentes a hora, minutos e segundos

% o template já faz no formato adequado
% eu fiz um arquivo .bib separado que é importado aqui
\bibliography{neurocomp}
% ---

% antes do apêndices um item opcional é o glossário
% se incluído é em ordem alfabética

% ----------------------------------------------------------
% Apêndices (opcional)
% ----------------------------------------------------------

% ---
% Inicia os apêndices
% ---
\begin{apendicesenv}
	
	% Imprime uma página indicando o início dos apêndices
	% a norma não diz para criar essa página
%	\partapendices

% usar um include com \chapter para cada apêndice
\chapter{Tutorial Python}\label{ap:python}
\section{Introdução}
\begin{itemize}
	\item Linguagem aberta, gratuita e de fácil aprendizado
	\item Linguagem de programação de alto nível (a programação é com palavras comuns da língua inglês, como \textit{print} para \textit{exibir} na tela)
	\item O uso no curso pode ser instalando o próprio pacote oficial do Python, instalando o Anaconda (um conjunto de ferramentas para processamento científico) ou usando o Google Colaboratory
	\item O \textit{Python} é uma linguagem interpretativa, significando que não é necessária a compilação para seu uso (não é necessário converter previamente o código para linguagem de máquina, como em outras linguagens). O interpretador do \textit{Python} pode ser executado interativamente, o que é útil para testar em tempo real a linguagem. Neste texto, códigos começados por \verb|>>>| são escritas diretamente no interpretador, e não em um arquivo de texto, e as linhas seguintes sem o \verb|>>>| são a resposta do comando executado. Linhas de continuação são representadas por \verb|...| e aparecem em construções de várias linhas, sempre endentadas (com um recuo, geralmente de 4 espaços em branco):
	\begin{minted}{python}
>>> print("Olá, mundo.")
Olá, mundo.
>>> if True:
...     print("Multi-linhas sempre precisam de endentação.")
... 
Multi-linhas sempre precisam de endentação.
	\end{minted}
\end{itemize}

\section{Comandos básicos}
\begin{itemize}
	\item Atribuições são feitas utilizando \verb|=| e comentários usando \verb|#|
	\begin{minted}{python}
# primeiro comentário
spam = 1  # segundo comentário
# um terceiro comentário
texto = "# este não é um comentário, pois está entre aspas."
	\end{minted}
	\item O interpretador funciona como uma simples calculadora, com os símbolos padrão funcionando para a maioria dos operadores:
	\begin{minted}{python}
>>> 2 + 2
4
>>> 50 - 5*6
20
>>> (50 - 5*6) / 4
5.0
>>> 8 / 5  # divisão sempre retorna um número em ponto flutuante (não inteiro)
1.6
	\end{minted}
	\item Para descartar a parte fracionária da divisão usa-se \verb|//| e \verb|%| para calcular o resto da divisão
	\begin{minted}{python}
>>> 17 / 3  # divisão clássica retorna ponto flutuante
5.666666666666667
>>>
>>> 17 // 3  # descarta a parte fracionária
5
>>> 17 % 3  # retorna o resto da divisão
2
>>> 5 * 3 + 2  # quociente * divisor + resto
17
	\end{minted}
	\item Usa-se \verb|**| para calcular potências
	\begin{minted}{python}
>>> 5 ** 2  # 5 ao quadrado
25
>>> 2 ** 7  # 2 elevado a 7
128
	\end{minted}
	\item Nenhum resultado é exibido ao fazer atribuições a uma variável
	\begin{minted}{python}
>>> largura = 20
>>> altura = 5 * 9
>>> largura * altura
900
	\end{minted}
\end{itemize}

\section{Sequências de texto (\textit{strings})}
\begin{itemize}
	\item Sequências de texto podem ser definidas entre aspas simples (\verb|'...'|) ou duplas (\verb|"..."|) com o mesmo resultado, e o caractere \verb|\| é usado para escapar as aspas (considerá-las como parte do texto, e não o marcador de fim da \textit{string}):
	\begin{minted}{python}
>>> 'ovo frito'  # aspas simples
'ovo frito'
>>> 'McDonald\'s' # usando \' para escapar a aspa simples
"McDonald's"
>>> "McDonald's"  # ... ou use aspas duplas
"McDonald's"
>>> '"Sim", eles disseram.'
'"Sim", eles disseram.'
>>> "\"Sim\", eles disseram."
'"Sim", eles disseram.'
	\end{minted}
	\item \textit{Strings} podem ser concatenadas (juntadas) usando o operador \verb|+|, e repetidas com \verb|*|
	\begin{minted}{python}
>>> # 'a' seguido de 'ra' duas vezes
>>> 'a' + 2 * 'ra'
'arara'
	\end{minted}
	\item \textit{Strings} podem ser indexadas, com o primeiro caractere tendo índice 0, e pode-se utilizar índices negativos, para contar da direita para a esquerda
	\begin{minted}{python}
>>> palavra = 'Python'
>>> palavra[0]  # caractere na posição 0
'P'
>>> palavra[5]  # caractere na posição 5
'n'
>>> palavra[-1]  # último caractere
'n'
>>> palavra[-2]  # penúltimo caractere
'o'
>>> palavra[-6]
'P'
	\end{minted}
	\item O fatiamento (\textit{slicing}) também é suportado. Enquanto a indexação é usado para o acesso a um único caractere, o fatiamento permite a obtenção de sub-sequências. O começo sempre é incluído, enquanto o final é sempre excluído, o que garante que \verb|s[:i] + s[i:]| é sempre igual à \verb|s|
	\begin{minted}{python}
>>> palavra[0:2]  # caractere da posição 0 (incluído) até 2 (excluído)
'Py'
>>> palavra[2:5]  # caractere da posição 2 (incluído) até 5 (excluído)
'tho'
>>> palavra[:2] + palavra[2:]
'Python'
>>> palavra[:4] + palavra[4:]
'Python'
	\end{minted}
	\item Os índices de fatiamento possuem alguns padrões úteis: a omissão do primeiro item tem 0 como padrão, enquanto que a omissão do segundo tem como padrão o comprimento da \textit{string}
	\begin{minted}{python}
>>> palavra[:2]  # caracteres do começo até a posição 2 (excluído)
'Py'
>>> palavra[4:]  # caracteres da posição 4 (incluído) até o final
'on'
>>> palavra[-2:]  # caracteres da penúltima posição (incluído) até o final
'on'
	\end{minted}
	\item A função \verb|len()| retorna o comprimento da string
	\begin{minted}{python}
>>> s = 'supercalifragilisticexpialidocious'
>>> len(s)
34
	\end{minted}
\end{itemize}

\section{Estruturas de controle de fluxo}
\begin{description}
	\item[enquanto (\textit{while})] Repete uma sequência de ações enquanto a condição for verdadeira
	\begin{minted}{python}
>>> # Sequência de Fibonacci:
... # a soma de dois elementos define o próximo
... a, b = 0, 1  # atribuição múltipla (a recebe 0, b recebe 1)
>>> while a < 10:  # a condição é o valor de a ser menor que 10
...     print(a)
...     a, b = b, a+b
...
0
1
1
2
3
5
8
	\end{minted}
	\item[se (\textit{if})] Executa uma ação se a condição for verdadeira. Pode ter inúmeras outras condições caso a primeira seja falsa (comando \verb|elif|) e uma caso todas sejam falsas (comando \verb|else|)
	\begin{minted}{python}
>>> x = int(input("Por favor, digite um inteiro: "))
Por favor, digite um inteiro: 42
>>> if x < 0:
...     x =  0
...     print('Valor negativo alterado para zero')
... elif x == 0:
...     print('Zero')
... elif x == 1:
...     print('Unitário')
... else:
...     print('Mais')
... 
Mais
	\end{minted}
	\item[para (\textit{for})] Itera ao longo dos itens de uma sequência
	\begin{minted}{python}
>>> # Mede algumas strings:
>>> palavras = ['gato', 'janela', 'marmota']
>>> for palavra in palavras:
...    print(palavra, len(palavra))
... 
gato 4
janela 6
marmota 7
	\end{minted}
	\item[faixa (\textit{range})] Gera progressões aritméticas. É útil para iterar ao longo de uma sequência de números. Os argumentos são, na ordem, \textbf{início} (primeiro valor da sequência, inclusivo), \textbf{fim} (valor final da sequência, exclusivo) e \textbf{passo} (de quanto em quanto os valores são contados). Caso tenha apenas dois argumentos, assume que o passo é 1, e caso só tenha um argumento, assume que o primeiro valor é 0
	\begin{minted}{python}
>>> for i in range(5):
...     print(i)
...
0
1
2
3
4
"""
range(5, 10)
5, 6, 7, 8, 9

range(0, 10, 3)
0, 3, 6, 9

range(-10, -100, -30)
-10, -40, -70
"""
	\end{minted}
\end{description}

\section{Funções}
\begin{itemize}
	\item Funções em \textit{Python} são definidas pela palavra-chave \verb|def|, seguida do nome da função e a lista de parâmetros entre parêntesis
	\item Todas as definições dentro da função precisam estar endentadas
	\item A primeira linha de definição pode ser uma sequência de texto, chamada de \textit{docstring}. Se presente, é usada para descrever a função e seu uso \footnote{Mais informações sobre \textit{docstrings} podem ser encontradas na \href{https://docs.python.org/3/tutorial/controlflow.html\#documentation-strings}{documentação online do \textit{Python}} (em inglês)}
	\begin{minted}{python}
>>> def fib(n):    # escreve a sequência de Fibonacci até n
...     """Escreve a sequência de Fibonacci até n."""
...     a, b = 0, 1
...     while a < n:
...         # o 'end' aqui indica que os valores serão separados por espaços
...         # ao invés de separados por linha (o padrão)
...         print(a, end=' ')
...         a, b = b, a+b
...     print()
...
>>> # Agora chama a função que acabou de ser definida:
... fib(2000)
0 1 1 2 3 5 8 13 21 34 55 89 144 233 377 610 987 1597
	\end{minted}
\end{itemize}

\section{Bibliotecas}
\begin{itemize}
	\item Bibliotecas (módulos) no \textit{Python} são importadas usando a sintaxe abaixo:
	\begin{minted}{python}
		import sys
	\end{minted}
	\item Podem ser importadas atribuindo-se um apelido:
	\begin{minted}{python}
		import builtins as bt
	\end{minted}
	\item Funções específicas podem ser importadas diretamente:
	\begin{minted}{python}
		import sound.effects.echo
		# ou
		from sound.effects import echo
	\end{minted}
	\item E também podem ter apelidos:
	\begin{minted}{python}
		import sound.effects.echo as see
		# ou
		from sound.effects import echo as see
	\end{minted}
	\item Todas as funções de um pacote podem ser importadas (não recomendável):
	\begin{minted}{python}
		from sound.effects import *
	\end{minted}
	\item Códigos de outros arquivos podem ser importados como módulos:
	\begin{minted}{python}
		import meu_arquivo
		# ou
		from meu_arquivo import minha_funcao
	\end{minted}
\end{itemize}

\subsection{Numpy}
\begin{itemize}
	\item Biblioteca para processamento matemático~\cite{harris_array_2020}
	\item Geralmente é importada usando \mint{python}|import numpy as np|
	\item O elemento base é o vetor (\textit{array}), que possui diversos atributos e métodos associados a ele
	\begin{minted}{python}
>>> a = np.array([2, 3, 4])
>>> a
array([2, 3, 4])
>>> a.shape
(3,)
>>> a.ndim
1
	\end{minted}
	\item É comum não se saber o conteúdo futuro de um vetor, mas o tamanho sim, e por isso existem funções para criar vetores com conteúdos e tamanhos específicos:
	\begin{description}
		\item[zeros] Cria um vetor preenchido com zeros
		\begin{minted}{python}
>>> np.zeros((3,4))
array([[0., 0., 0., 0.],
[0., 0., 0., 0.],
[0., 0., 0., 0.]])
		\end{minted}
		\item[ones] Cria um vetor preenchido com 1
		\begin{minted}{python}
>>> np.ones((2,3))
array([[1., 1., 1.],
[1., 1., 1.]])
		\end{minted}
		\item[empty] Cria um vetor com valores aleatórios extremamente pequenos, obtidos com base no que está presente na memória, e por isso é ligeiramente mais rápido que os anteriores para ser gerado
		\begin{minted}{python}
>>> np.empty((2, 4))
array([[4.67379042e-310, 0.00000000e+000, 6.79038654e-313,
2.22809558e-312],
[2.14321575e-312, 2.46151512e-312, 2.41907520e-312,
5.97819431e-322]])
		\end{minted}
		\item[arange] Cria um vetor com uma sequência de números com um passo
		\begin{minted}{python}
>>> np.arange(10, 30, 5) # argumentos: inicio, fim, passo
array([10, 15, 20, 25])
		\end{minted}
		\item[linspace] Cria um vetor com uma sequência de números com tamanho específico
		\begin{minted}{python}
>>> np.linspace(0, 2, 9) # 9 números de 0 até 2
array([0.  , 0.25, 0.5 , 0.75, 1.  , 1.25, 1.5 , 1.75, 2.  ])
		\end{minted}
	\end{description}
	\item O pacote também contém funções matemáticas comuns ($sin$, $cos$, $exp$, $sqrt$, etc)
	\begin{minted}{python}
>>> B = np.arange(3)
>>> B
array([0, 1, 2])
>>> np.exp(B)
array([1.        , 2.71828183, 7.3890561 ])
>>> np.sqrt(B)
array([0.        , 1.        , 1.41421356])
>>> C = np.array([2., -1., 4.])
>>> np.add(B, C)
array([2., 0., 6.])
	\end{minted}
	\item Indexação e fatiamento (\textit{slices}) também são possíveis em vetores (semelhante à listas):
	\begin{minted}{python}
>>> a = np.arange(10)**3
>>> a
array([  0,   1,   8,  27,  64, 125, 216, 343, 512, 729])
>>> a[2]
8
>>> a[2:5]
array([ 8, 27, 64])
>>> # equivalente a a[0:6:2] = 1000;
>>> # do início até a posição 6, exclusivo, define cada segundo elemento para 1000
>>> a[:6:2] = 1000
>>> a
array([1000,    1, 1000,   27, 1000,  125,  216,  343,  512,  729])
>>> a[::-1]  # a reverso
array([ 729,  512,  343,  216,  125, 1000,   27, 1000,    1, 1000])
>>> for i in a:
...     print(i**(1 / 3.))
...
9.999999999999998
1.0
9.999999999999998
3.0
9.999999999999998
4.999999999999999
5.999999999999999
6.999999999999999
7.999999999999999
8.999999999999998
	\end{minted}
	\item Mais comandos e exemplos do \textit{Numpy} estão na \href{https://numpy.org/doc/stable/}{documentação online} (em inglês)
\end{itemize}

\subsection{Matplotlib}
\begin{itemize}
	\item Biblioteca para gerar visualizações estáticas, animadas ou interativas (gráficos)~\cite{hunter_matplotlib_2007}
	\item Geralmente é importada usando \mint{python}|import matplotlib.pyplot as plt| frequentemente junto com a \textit{Numpy}
	\item O elemento base é a figura (\textit{Figure}), cada uma contendo um ou mais eixos (\textit{Axes}), isto é, a área onde os pontos podem ser especificados em termos de coordenadas x-y, theta-r, x-y-z, etc.
	\begin{minted}{python}
import matplotlib.pyplot as plt
import numpy as np
fig, ax = plt.subplots() # cria uma figura contendo um único eixo
ax.plot([1, 2, 3, 4], [1, 4, 2, 3] # exibe alguns dados no eixo
	\end{minted}
	
	\begin{figure}[htb!]
		\centering
		\caption{Exemplo simples de plot}
		\label{fig:plot}
		\includegraphics[width=0.8\textwidth]{figs/plot.png}
		\fonte{O autor (\the\year)}
	\end{figure}
	
	\item Tipos de gráficos comuns:
	\begin{description}
		\item[plot] Plota $x$ vs $y$ como linhas e/ou marcadores
		\begin{minted}{python}
>>> plot(x, y)        # plota x e y usado tipo padrão de linha e cor
>>> plot(x, y, 'bo')  # plota x e y usando círculos azuis como marcadores
>>> plot(y)           # plota y usando x como um vetor de 0 até N-1
>>> plot(y, 'r+')     # igual acima, mas com + vermelhos de marcador
		\end{minted}
		\item[stem] Plota linhas perpendiculares em relação à linha de base, começando na linha de base e terminando no topo, e coloca um marca nesse lugar. Normalmente, $x$ são as posiçoes e $y$ os topos.
		\begin{minted}{python}
x = np.linspace(0.1, 2 * np.pi, 41)
y = np.exp(np.sin(x))
plt.stem(x, y)
plt.show()
		\end{minted}
		
		\begin{figure}[htb!]
			\centering
			\caption{Gráfico do tipo Stem}
			\label{fig:stem}
			\includegraphics[width=0.8\textwidth]{figs/stem.pdf}
			\fonte{O autor (\the\year)}
		\end{figure}
		\item[scatter] Plota y vs x com diferentes tamanhos e cores dos marcadores
		\begin{minted}{python}
# fixa a semente de números aleatórios
np.random.seed(19680801)

N = 50
x = np.random.rand(N)
y = np.random.rand(N)
colors = np.random.rand(N)
area = (30 * np.random.rand(N))**2  # 0 até 15 pontos o tamanho do raio

plt.scatter(x, y, s=area, c=colors, alpha=0.5)
plt.show()
		\end{minted}
		
		\begin{figure}[htb!]
			\centering
			\caption{Gráfico de dispersão (scatter)}
			\label{fig:scatter}
			\includegraphics[width=0.8\textwidth]{figs/scatter.pdf}
			\fonte{O autor (\the\year)}
		\end{figure}
	\end{description}
	\item Mais comandos e exemplos do \textit{Matplotlib} estão na \href{https://matplotlib.org/stable/contents.html}{documentação online} (em inglês)
\end{itemize}

\section{Dicas gerais}
\begin{itemize}
	\item Use 4 espaços para endentação, e não tabulações (alguns editores de texto convertem automaticamente as tabulações para 4 espaços, vale verificar)
	\item Nomeie as variáveis de acordo com o que elas se referem de maneira explícita (usar \verb|area| ao invés de \verb|a|, por exemplo)
	\item Nomeie variáveis e funções com letras minúsculas e palavras separadas por espaço (esse método é chamado de \verb|snake_case|)
	\item Escreva linhas com até 79 caracteres, e até 72 se for linha com comentário (a maioria dos editores de texto exibe o número de caracteres)
	\item Use linhas em branco para separar funções ou longos blocos de texto dentro de funções
	\item Quando possível, coloque o comentário na linha a que se refere (respeitando o limite de caracteres citado acima)
	\item Use \textit{docstrings} para arquivos e funções, descrevendo de maneira adequada o uso
	\item Use espaços entre operadores e depois de vírgulas (ex.: \verb|a = f(1, 2) + g(3, 4)|)
	\item Mais comandos e recursos do \textit{Python} podem ser encontrados no \href{https://docs.python.org/3/tutorial/index.html}{tutorial online} (em inglês), e dicas de estilo (como deixar o código "bonito" para o \textit{Python}) estão no documento chamado \href{https://www.python.org/dev/peps/pep-0008}{PEP 8} (em inglês)
\end{itemize}

	
\end{apendicesenv}
% ---

% ----------------------------------------------------------
% Anexos (opcional)
% ----------------------------------------------------------

% ---
% Inicia os anexos
% ---
%\begin{anexosenv}
	
	% Imprime uma página indicando o início dos anexos
	% novamente a norma não diz para criar essa página
%	\partanexos

% novamente usar um include com \chapter para cada anexo
% ou usar \includepdf caso seja documento externo (o mais provável no caso de anexo)
	
%\end{anexosenv}

% o último item opcional é o índice
% se incluído é conforme a NBR 6034

\end{document}
